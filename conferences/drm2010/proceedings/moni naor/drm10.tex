\documentclass{sig-alternate}


\def\draft{0}   % 1 for working draft

\ifnum\draft=1 % show authors' note if draft
    \def\ShowAuthNotes{1}
\else
    \def\ShowAuthNotes{0}
\fi

%%%%%%%%%%%%%%%%%%%%%%%%%%%%%%%%
% Author's notes
\ifnum\ShowAuthNotes=1
\newcommand{\authnote}[2]{{ \footnotesize \bf{\color{Red}[#1's Note: {\color{Blue}#2}]}}}
\else
\newcommand{\authnote}[2]{}
\fi

\newcommand{\cnote}[1]{{\authnote{Cynthia} {#1}}}
\newcommand{\gnote}[1]{{\authnote{Guy} {#1}}}
\newcommand{\mnote}[1]{{\authnote{Moni} {#1}}}
\newcommand{\onote}[1]{{\authnote{Omer} {#1}}}
\newcommand{\snote}[1]{{\authnote{Salil} {#1}}}

\newtheorem{theorem}{Theorem}[section]
\newtheorem{corollary}[theorem]{Corollary}
\newtheorem{remark}{Remark}[section]

\newtheorem{lemma}[theorem]{Lemma}%[section]
\newtheorem{proposition}{Proposition}[section]
\newtheorem{claim}[theorem]{Claim}%[section]
\newtheorem{fact}{Fact}[section]

\newtheorem{definition}{Definition}[section]

\newcommand{\C}{\mathcal C}
\newcommand{\U}{\mathcal U}
\newcommand{\M}{\mathcal M}
\newcommand{\T}{\mathcal T}
\newcommand{\SM}{\mathcal C}
\newcommand{\A}{\mathcal A}
\newcommand{\B}{\mathcal B}
\newcommand{\V}{\mathcal V}
\newcommand{\D}{\mathcal D}
\newcommand{\E}{\mathcal E}
\newcommand{\F}{\mathcal F}
\newcommand{\Hw}{\mathcal H}
\newcommand{\Orc}{\mathcal O}
\newcommand{\Prog}{\mathcal P}
\newcommand{\Adv}{\mathcal A}
\newcommand{\Rand}{\mathcal R}
\newcommand{\Obf}{\mathcal O}
\newcommand{\Sim}{\mathcal S}
\newcommand{\Protocol}{\mathcal P}
\newcommand{\Le}{\mathcal L}
\newcommand{\Pre}{\mathcal P}
\newcommand{\Sam}{\mathcal S}
\newcommand{\G}{\mathcal G}
\newcommand{\Hash}{\mathcal H}
\newcommand{\Rt}{\mathbb{R}}
\newcommand{\Nt}{\mathbb{N}}
\newcommand{\grad}{\nabla}
\newcommand{\secp}{\kappa}
\newcommand{\sbd}{s_{bad}}
\newcommand{\Dstb}{\mathcal D}
\newcommand{\DB}{D}



\newcommand{\remove}[1]{}

\def\zo{\{0,1\}}

\newcommand{\Zn}{}
\newcommand{\Z}[1]{\mathbb{Z}^*_{#1}}
\newcommand{\eps}{\varepsilon}
%\newcommand{\prob}{{\rm Pr}}
\newcommand{\pr}[2]{\prob_{#1}\left[ #2\right]}
\newcommand{\poly}{{\rm poly}}
\newcommand{\com}{{\rm com}}
\newcommand{\negl}{{\rm negl}}
\newcommand{\NP}{{\mbox{NP}}}
\newcommand{\BPP}{{\cal BPP}}
\newcommand{\polylog}{{\rm polylog}}

\newcommand{\Lap}{\mathit{Lap}}
\newcommand{\Gen}{\mathit{Gen}}
\newcommand{\Sign}{\mathit{Sign}}
\newcommand{\Verify}{\mathit{Verify}}
\newcommand{\Setup}{\mathit{Setup}}
\newcommand{\Enc}{\mathit{Enc}}
\newcommand{\Dec}{\mathit{Dec}}
\newcommand{\Trace}{\mathit{Trace}}
\newcommand{\Encrypt}{\mathit{Encrypt}}
\newcommand{\Decrypt}{\mathit{Decrypt}}
\newcommand{\mysmall}{\mathit{small}}

\newcommand{\myparagraph}[1]{\medskip \noindent \textbf{#1}}

\newcommand\op[1]{}

\begin{document}
\conferenceinfo{DRM'10,} {Oct 4, 2010, Chicago, Illinois,
USA.} \CopyrightYear{2010} \crdata{978-1-60558-506-2/09/05}

\title{The Privacy of Tracing Traitors}



\numberofauthors{1}
\author{
\alignauthor Moni Naor\titlenote{Incumbent of the Judith Kleeman Professorial Chair, Department of Computer
Science and Applied Mathematics, Weizmann Institute of Science,
Rehovot 76100, Israel. E-mail: \texttt{moni.naor@weizmann.ac.il}.
Research supported in
part by a grant from the Israel Science Foundation.}\\
\affaddr{Weizmann Institute of Science}
}


\date{}

\maketitle


In this talk I will explore a connection between {\bf traitor tracing} schemes and the problem of {\bf sanitizing} data to remove personal information 
while allowing statistically meaningful information to be released. It is based on joint work with Cynthia Dwork, Omer Reingold, Guy N. Rothblum and Salil Vadhan~\cite{DNRRV}

{\bf Traitor tracing} schemes are  method of content
distribution which  enable a publisher to trace a pirate
decryption box to a secret key (of a treacherous user) that was used
to create the box.  For example, consider a content provider who
wishes to broadcast his content to a group of subscribers. To do
this, he gives each of the subscribers a decryption box that uses a
secret key. Unfortunately, a subscriber might now build and
distribute a pirate decoder. Traitor tracing schemes ensure that if
such a pirate decoding box is found, the content distributer can run
a {\em tracing} algorithm to recover at least one private key used
to create the box, and perhaps try to take legal action against this
subscriber. Tracing schemes were first formally introduced by Chor, Fiat and
Naor~\cite{ChorFN94} and there are many constructions with various
choices of the parameters (see \cite{BonehN08} for recent
references).

Consider {\bf private data analysis} in the setting in which a trusted
and trustworthy curator, having obtained a large data set containing
private information, releases to the public a ``sanitization'' of the
data set that simultaneously protects the privacy of the individual
contributors of data and offers utility to the data analyst. 
The sanitization may be in the form of an arbitrary data structure,
accompanied by a computational procedure for determining approximate
answers to queries on the original data set. Alternatively it may be in the 
form of a ``synthetic data set'' consisting of data items drawn from the same
universe as items in the original data set; queries are carried out as
if the synthetic data set were the actual input.  

The ``gold standard" of privacy that has emerged in recent years is that of {\em differential privacy}:
for any two neighboring data sets, one containing information about an individual and the other one not containing it,
for any possible string released by the curator it should be difficult distinguish which of the two possible data sets yielded that response.

Recently there have been several powerful results showing that it is possible to sanitize
data sets while allowing relatively accurate answers to queries.  
For instance, Blum, Ligett and Roth (STOC `08) showed the following:
Let $X$ be a universe of data items and
$\C$ be a ``concept'' class consisting of efficiently computable
functions $f : X\rightarrow \{0,1\}$. 
Given a database $x\in X^n$, it is possible to obtain a {\em synthetic database} that remarkably
maintains approximately correct fractional counts for {\em all} concepts in $\C$, while ensuring a strong privacy guarantee in the above sense. 

The downside is that resulting mechanism is {\em not}, in general,  
computationally efficient and the natural question is whether it possible to obtain efficient solutions.
It turns out that this question has deep connections to cryptography. 
When either $|\C|$ or $|X|$ are superpolynomial and assuming
the existence of one-way functions there exist
distributions on data sets and a choice of $\C$ for which there is no efficient private
construction of a {\em synthetic} data set maintaining approximately correct
counts.

Turning to the potentially easier problem of privately generating a
data structure from which it is possible to approximate counts, there
is a tight connection between hardness of sanitization and the
existence of {\em traitor tracing} schemes. Using the scheme of Boneh, Sahai and Waters \cite{BonehSW06}, and under the 
appropriate  complexity assumptions,  it is possible
to obtain a distribution on
databases and a concept class permitting {\em inefficient} sanitization, but for which {\em no efficient sanitization is
possible}.

\begin{thebibliography}{10}
%MN1 added a small skip to look better
\vskip 0.1in
\bibitem{BlumLR08}
A.~Blum, K.~Ligett, and A.~Roth.
\newblock A learning theory approach to non-interactive database privacy.
\newblock In {\em STOC}, pages 609--618, 2008.

\bibitem{BonehN08}
D.~Boneh and M.~Naor.
\newblock Traitor tracing with constant size ciphertext.
\newblock In {\em ACM Conference on Computer and Communications Security},
  pages 501--510, 2008.

\bibitem{BonehSW06}
D.~Boneh, A.~Sahai, and B.~Waters.
\newblock Fully collusion resistant traitor tracing with short ciphertexts and
  private keys.
\newblock In {\em EUROCRYPT}, pages 573--592, 2006.

\bibitem{ChorFN94}
B.~Chor, A.~Fiat, and M.~Naor.
\newblock Tracing traitors.
\newblock In {\em CRYPTO}, pages 257--270, 1994.

\bibitem{DNRRV}
C.~Dwork, O.\~Reingold, G.~N.~Rothblum and S.~Vadhan, 
\newblock On the Complexity of Differentially Private Data Release
\newblock In {\em STOC}, pages 381--390, 2009.

\end{thebibliography}




\end{document}