\documentclass{article}

\usepackage{hyperref,url}
\usepackage{graphicx,color}
\usepackage{tabularx}
\usepackage{a4wide}

\setlength{\headheight}{0pt}
\setlength{\topmargin}{0pt}
\setlength{\headsep}{0pt}
\setlength{\footskip}{0pt}
\setlength{\oddsidemargin}{0pt}
\setlength{\marginparwidth}{0pt}
\setlength{\marginparsep}{0pt}
\addtolength{\textwidth}{31pt}
\hoffset-.5in
\addtolength{\textwidth}{1in}
\voffset-.7in
\addtolength{\textheight}{1.4in}


%\renewcommand{\tabularxcolumn}[1]{b{#1}}

\pagestyle{empty}


\begin{document}

\begin{center}
\vbox to 0pt{\hbox to0pt{\hskip10pt\includegraphics[width=.12\hsize]{sigsac.png}\hss}\vss}
  \textcolor[rgb]{.5,.5,.5}{\Huge Eighth ACM DRM Workshop}\\[-2mm]
  {\LARGE\bfseries \textcolor[rgb]{0,.08,.4}{Eighth ACM DRM Workshop~~}}\\[2mm]
  October 27, 2008\qquad{\tiny$^\bullet$}\qquad Alexandria, Virginia, USA
\end{center}
\begin{flushright}
  \textit{(co-located with ACM-CCS 2008)}
\end{flushright}\par\smallskip

\noindent
\begin{tabularx}{\linewidth}%{@{}p{.2\textwidth}p{.8\textwidth}@{}}
  {>{\small\setlength{\hsize}{.7\hsize}}X|%
    >{\setlength{\hsize}{1.3\hsize}}X}
%
%  \mbox{}\par
  {\large\bfseries Program chairs}\par\medskip
  \begin{tabular}{l}
    Gregory Heileman (U. New Mexico, USA)\\
    Marc Joye (Thomson, France)
  \end{tabular}\par\bigskip
  
  {\large\bfseries Program committee}\par\medskip
  \begin{tabular}{l}
    Olivier Billet (Orange Labs, France)\\
    Xavier Boyen (Voltage, USA)\\
    Alain Durand (Thomson, France)\\
    Rudiger Grimm (U. Koblenz, Germany)\\
    Bill Horne (Hewlett-Packard, USA)\\
    Hongxia Jin (IBM, USA)\\
    Aggelos Kiayias (U. Connecticut, USA)\\
    David Kravitz (Motorola Labs, USA)\\
    Brian LaMacchia (Microsoft, USA)\\
    William Lehr (MIT, USA)\\
    Nasir Memon (Polytechnic U., USA)\\
    Fernando Perez-Gonzalez (U. Vigo, Spain)\\
    Rei Safavi-Naini (U. Calgary, Canada)\\
    Bin Zhu (Microsoft, China)
  \end{tabular}\par\bigskip

  {\large\bfseries General chair}\par\medskip
  \begin{tabular}{l}
    Peng Ning (NCSU, USA)\\
    \tiny\itshape (also General chair for ACM-CCS~2008)
  \end{tabular}\par\bigskip

  {\large\bfseries Steering committee}\par\medskip
  \begin{tabular}{l}
    Joan Feigenbaum (Yale U., USA)\\
    Aggelos Kiayias (U. Connecticut, USA)\\
    Rei Safavi-Naini (U. Calgary, Canada)\\
    Tomas Sander (Hewlett-Packard, USA)\\
    Moti Yung (Google \& Columbia U., USA)
  \end{tabular}\par\vskip.9in

  {\large\bfseries Important dates}\par\medskip
  \begin{tabular}{l}
    Submission deadline: \textcolor[rgb]{.75,0,0}{\bfseries June 5, 2008}\\
    Notification of acceptance: July 20, 2008\\
    Camera-ready version: August 8, 2008\\
    Workshop: October 27, 2008
  \end{tabular}\par\vskip.9in

%   {\footnotesize ACM-DRM~2008 gratefully acknowledges the financial
%     support of:}\par
  {\footnotesize Thanks to the financial support of our sponsors a
    small amount of fellowships will be available to full-time
    students presenting papers.}\par\medskip
  {\centering
  \begin{tabular}{cc}
    {\includegraphics[width=.35\hsize]{microsoft.png}}&
    \raisebox{-10pt}{\includegraphics[width=.45\hsize]{thomson.png}}
  \end{tabular}}

&
  The ACM Workshop on Digital Rights Management is an international
  forum that serves as an interdisplinary bridge between areas that
  can be applied to solving the problem of Intellectual Property
  protection of digital content. These include: cryptography, software
  and computer systems design, trusted computing, information and
  signal processing, intellectual property law, policy-making, as well
  as business analysis and economics. Its purpose is to bring together
  researchers from the above fields for a full day of formal talks and
  informal discussions, covering new results that will spur new
  investigations regarding the foundations and practices of
  DRM.\par\medskip 

  This year's workshop, the eighth in the series, continues this
  tradition. As in the previous editions, it is sponsored by ACM
  SIGSAC and is held in conjunction with the ACM Conference in
  Computer and Communications Security (CCS).\par\medskip

  Topics of interest include but are not limited to:\par\smallskip
  {\small
  \begin{tabular}{l@{\thinspace}p{.92\hsize}}
    $-$ & anonymous publishing, privacy and DRM\\
    $-$ & architectures for DRM systems\\
    $-$ & business models for online content distribution, risk management\\
    $-$ & copyright-law issues, including but not limited to fair use\\
    $-$ & digital goods and online multiplayer games\\
    $-$ & digital policy management\\
    $-$ & DRM and consumer rights, labeling and competition law\\
    $-$ & implementations and case studies\\
    $-$ & information theory and combinatorics, including marking
    assumptions and related codes\\
    $-$ & robust identification of digital content\\
    $-$ & security issues, including but not limited to authorization,
    encryption, tamper resistance, and watermarking\\
    $-$ & regulatory authority for DRM, interoperability\\
    $-$ & supporting cryptographic technology including but not limited to
    traitor tracing, broadcast encryption, obfuscation\\
    $-$ & threat and vulnerability assessment\\
    $-$ & trusted computing, attestation, hardware support for DRM,
    side-channels\\ 
    $-$ & usability aspects of DRM systems\\
    $-$ & web services related to DRM systems
  \end{tabular}}


  \section*{Instructions for authors}

  Submissions must not overlap with papers that have been published or
  that are simultaneously submitted to a journal or a conference with
  proceedings. Submissions should be at most 15 pages excluding the
  bibliography and well-marked appendices, using at least 11-point
  font and reasonable margins. Committee members are not required to
  read the appendices, and thus submissions should be intelligible
  without them. Each submission should start with the title, abstract,
  and names and contact information of authors. All submissions will
  be handled electronically.  For submission instructions and further
  information please point your web-browser to:
  \begin{center}
    \url{http://www.ece.unm.edu/DRM2008/}
  \end{center}
  
  \section*{Proceedings}

  Accepted papers will be published in an archival proceedings volume
  by ACM Press and will be distributed at the time of the workshop.
\end{tabularx}

\end{document}
