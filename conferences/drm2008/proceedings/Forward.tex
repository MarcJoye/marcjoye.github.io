\documentclass[11pt]{article}
\oddsidemargin 0in
\evensidemargin 0in
\topmargin -1in
\textwidth 6.5in
\textheight 9in
\parindent 0in

\begin{document}

\begin{center}
{\bf\Large Foreward}
\end{center}

\noindent
For as long as creative materials have been produced, the owners of those materials have sought to protect their rights over them. Indeed, approximately 500 years ago, Miguel de Cervantes had to deal with impostors seeking to publish sequels to the first part of his wildly popular {\it Man of LaMancha}.  In the present day, the same problems exist, further exacerbated by the ongoing ``digital revolution". The field of digital rights management~(DRM) has formed around those seeking to deal with the modern instantiation of these problems. Furthermore, DRM~(contrary to reports of its untimely death) is not going away --- as long as content owners have concerns about their rights, DRM by definition will exist. The real question, then, is not ``should DRM exist or not?", but rather ``in what form will DRM exist?", and ``in what manner will it evolve?". This volume, the eighth in a series associated with the ACM DRM Workhsops, contains research contributions by those at the forefront of the field seeking answers to the latter two questions.~\\

The workshop itself, held in conjunction with the 15th ACM Conference on Computer and Communications Security (CSS), consisted of talks and discussions associated with each of the papers contained in this volume, two invited talks, and a panel discussion involving participants with widely differing views on the future direction of DRM. One of the invited talks was given by Internet pioneer Robert Kahn, and considered naming infrastructures associated with past and emerging DRM systems, while the other was given by cryptographer Yacov Yacobi, and considered content identification as well as the problem of counterfeiting software and content.  During the panel discussion, recognized experts in the field considered the past development of DRM, with is many false starts, and the likely direction of future DRM usages.  The panel included representatives from industry involved with the deployment of commercial DRM systems, as well as representatives from academia that have wrestled with the difficult policy and consumer rights issues connected with the field.
~\\

We would like to thank all of the authors who submitted papers to this year's workshop, as well as the program committee members and external reviewers who participated in the review process.  Furthermore, we would like to thank Pramod Jamkhedkar who was our system administrator at the University of New Mexico where the workshop webpage was hosted and the submission server was maintained.  In addition, we are grateful for the support of the ACM SIGSAC, Microsoft, and Thomson for hosting and sponsoring the workshop.  Finally, we acknowledge the hard work of Lisa M. Tolles at Sheridan Printing, who produced this proceedings volume. 
\begin{tabbing}

{\it ACM-DRM'08}  \ \ \ \ \ \ \ \ \ \ \ \= {\bf Gregory L. Heileman} \ \ \ \ \ \ \ \ \ \ \= {\bf Marc Joye} \\
{\it Proceedings Editors} \> University of New Mexico \>  Thomson R\&D France \\
{\it co-Program Chairs}
\end{tabbing}
\end{document}

